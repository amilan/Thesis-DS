\pagenumbering{roman} 
\setcounter{page}{1} 
\pagestyle{plain}

%%%%%%%%%%%%%%%%
%%% CREDITOS %%%
%%%%%%%%%%%%%%%%
\chapter*{Créditos/Copyright}

Una página con la especificación de créditos/copyright para el proyecto (ya sea aplicación por un lado y documentación por el otro, o unificadamente), así como la del uso de marcas, productos o servicios de terceros (incluidos códigos fuente). Si una persona diferente al autor colaboró en el proyecto, tiene que quedar explicitada su identidad y qué hizo.

A continuación se ejemplifica el caso más habitual, aunque se puede modificar por cualquier otra alternativa:

\vspace{1cm}

\begin{figure}[ht]
    \centering
	\includegraphics[scale=1]{../images/license.png}
\end{figure}

Esta obra está sujeta a una licencia de Reconocimiento -  NoComercial - SinObraDerivada

\href{https://creativecommons.org/licenses/by-nc-nd/3.0/es/}{3.0 España de CreativeCommons}.

%%%%%%%%%%%%%
%%% FICHA %%%
%%%%%%%%%%%%%
\chapter*{THESIS INDEX CARD}

\begin{table}[ht]
	\centering{}
	\renewcommand{\arraystretch}{2}
	\begin{tabular}{r | p{8cm}}
		\hline
		Title: & Study of the transcriptional function \newline of Cyclin D1 in Leukemia\\
		\hline
        Author: & Antonio Milán Otero\\
		\hline
        Nombre del colaborador/a docente: & Carles Barceló\\
		\hline
        Nombre del PRA (PRA???): & Jordi Casas Roma\\
		\hline
        Date of delivery (mm/aaaa): & 06/2019\\
		\hline
        Degree or program: & MSc in Data Science\\
		\hline
        Thesis area: & Data Mining and Machine Learning\\
		\hline
        Language: & English\\
		\hline
        Keywords & Leukemia, transcriptional function, cyclin D1, machine learning\\
		\hline
	\end{tabular}
\end{table}

%%%%%%%%%%%%%%%%%%%
%%% DEDICATORIA %%%
%%%%%%%%%%%%%%%%%%%
\chapter*{Dedication/Quote}

Breves palabras de dedicatoria y/o una cita.

%%%%%%%%%%%%%%%%%%%
%%% Agradecimientos %%%
%%%%%%%%%%%%%%%%%%%
\chapter*{Acknowledgment}

Si se considera oportuno, mencionar a las personas, empresas o instituciones que hayan contribuido en la realización de este proyecto.

%%%%%%%%%%%%%%%%
%%% RESUMEN  %%%
%%%%%%%%%%%%%%%%
\chapter*{Abstract}
\addcontentsline{toc}{chapter}{Abstract}

\onehalfspacing

Texto con la síntesis del proyecto, esto es, un texto en el cual se explica de manera concisa la definición del proyecto/problema abordado, sus objetivos/métodos de resolución, y los resultados y conclusiones (no puede ser una lista, sino un texto continuo redactado de manera estructurada). Si es necesario poner una referencia en este texto, ésta será anotada a pie de la misma página. En este apartado se puede usar un lenguaje más literario y coloquial que para el resto del documento.

El Abstract se escribirá por duplicado. Una de las versiones tiene que ser \textbf{obligatoriamente en inglés}. La otra versión tiene que estar escrita en catalán o español. En caso de no escribir el resto del documento en inglés, será necesario escribir la segunda versión del Abstract en el idioma utilizado para el resto de la memoria. La palabra Abstract se cambiará por ``\textbf{Resum}'' o ``\textbf{Resumen}'' en la versión catalana y española, respectivamente. 

Extensión recomendada: 250 palabras máximo.

Como escribir un buen Abstract (en inglés):

\href{http://www.ece.cmu.edu/~koopman/essays/abstract.html}{http://www.ece.cmu.edu/~koopman/essays/abstract.html}

\vspace{1.5cm}

\textbf{Palabras clave}: Leukemia, transcriptional function, cyclin D1, machine learning