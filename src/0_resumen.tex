\pagenumbering{roman} 
\setcounter{page}{1} 
\pagestyle{plain}

%%%%%%%%%%%%%%%%
%%% CREDITOS %%%
%%%%%%%%%%%%%%%%
\chapter*{License}

% Una página con la especificación de créditos/copyright para el proyecto (ya sea aplicación por un lado y documentación por el otro, o unificadamente), así como la del uso de marcas, productos o servicios de terceros (incluidos códigos fuente). Si una persona diferente al autor colaboró en el proyecto, tiene que quedar explicitada su identidad y qué hizo.

% A continuación se ejemplifica el caso más habitual, aunque se puede modificar por cualquier otra alternativa:

\vspace{1cm}

\begin{figure}[ht]
    \centering
	\includegraphics[scale=1]{../images/license_int.png}
\end{figure}

This work is licensed under a Creative Commons Attribution - NonCommercial - NoDerivatives 4.0 International License.

\href{https://creativecommons.org/licenses/by-nc-nd/4.0/}{Creative Commons 4.0 International License}.

% Esta obra está sujeta a una licencia de Reconocimiento -  NoComercial - SinObraDerivada

% \href{https://creativecommons.org/licenses/by-nc-nd/3.0/es/}{3.0 España de CreativeCommons}.

%%%%%%%%%%%%%
%%% FICHA %%%
%%%%%%%%%%%%%
\chapter*{THESIS INDEX CARD}

\begin{table}[ht]
	\centering{}
	\renewcommand{\arraystretch}{2}
	\begin{tabular}{r | p{8cm}}
		\hline
		Title: & Study of the transcriptional function \newline of Cyclin D1 in Leukemia\\
		\hline
        Author: & Antonio Milán Otero\\
		\hline
        Teacher collaborator: & Carles Barceló\\
		\hline
        Teacher responsible for the subject: & Jordi Casas Roma\\
		\hline
        Date of delivery (mm/aaaa): & 06/2019\\
		\hline
        Degree or program: & MSc in Data Science\\
		\hline
        Thesis area: & Data Mining and Machine Learning\\
		\hline
        Language: & English\\
		\hline
        Keywords & Leukemia, Mantle Cell Lymphoma, Cyclin D1, DNA-damage Repair, Machine Learning\\
		\hline
	\end{tabular}
\end{table}

%%%%%%%%%%%%%%%%%%%
%%% DEDICATORIA %%%
%%%%%%%%%%%%%%%%%%%
\chapter*{Dedication/Quote}

To be completed.

%%%%%%%%%%%%%%%%%%%
%%% Agradecimientos %%%
%%%%%%%%%%%%%%%%%%%
\chapter*{Acknowledgment}

To be completed.

%%%%%%%%%%%%%%%%
%%% RESUMEN  %%%
%%%%%%%%%%%%%%%%
\chapter*{Abstract}
\addcontentsline{toc}{chapter}{Abstract}

\onehalfspacing

%%Motivation
%% Why?
Leukemia is a type of cancer that starts in blood-forming tissue, such as the bone marrow. It causes the production of large numbers of abnormal blood cells that end up entering into the bloodstream. \footnote{https://www.cancer.gov/publications/dictionaries/cancer-terms/def/leukemia}

Within the different types of leukemia, Mantle Cell Lymphoma (MCL) is the one with the worst prognosis due to the short survival average of a patient, which is close to three years.
This tumor is characterized by the over-expression of Cyclin D1, a protein that helps control cell division. MCL is also characterized by the binding of this protein to certain regions of DNA involved in the regulation of DNA-damage response (DDR).

%% Problem statement
%% What problem do we try to solve? 
%% Add the explanation of DDR!!!!

The presented study aims to identify similarities between the gene expression regulated by Cyclin D1 in lymphomas and the gene expression in DNA damage.
That knowledge will allow the exploration of essential mechanisms of carcinogenesis and help in the identification of genes that could be an interesting therapeutic target in the process of tumor progression. Additionally, new biomarkers that could be used in early diagnosis can be found.

%% Approach
%% How did you go about solving or making progress on the problem?
With the addition of Machine Learning algorithms to the biology analysis pipeline, this project explores new ways to improve the traditional methodologies and boost the identification of significantly enriched genes that will serve the purposes mentioned above.

%% Results
%% What's the answer?
%% To be filled lated.

%% Conclusions
%% What are the implications of your answer?
%% To be filled later

\vspace{1.5cm}

\textbf{Keywords}: Leukemia, Mantle Cell Lymphoma, Cyclin D1, DNA-damage Repair, Machine Learning.