% \chapter{Literature Review}
% \label{chapter:literature}

%%% SECTION
\section{Literature Review}
\subsection{Foundations}

The overexpression of cyclin D1 in human cancer is well known\cite{Lamb2003} and has been reported in several studies\cite{2017Reena}.

An interesting recent study conducted by Albero et al.\cite{10.1172/JCI96520}, focuses on the study of this overexpression and how it produces a global trancriptional donwmodulation in lymphoid neoplasms. In their own words:

\textit{This finding of global transcriptional dysregulation expands the known functions of oncogenic cyclin D1 and suggests the therapeutic potential of targeting the transcriptional machinery in cyclin D1–overexpressing tumors.}
\cite{10.1172/JCI96520}

In parallel, other studies \cite{DiSante2017} \cite{Casimiro2016}, have shown the role of cyclin D1 in the cell cycle and its influence in the DNA-damage repair process.

This two concepts have triggered the main idea behind the objectives of the project presented here.

In addition to the already commented works, the field of artificial intelligence and in particular the machine learning discipline inside of it, has been winning attention in many fields, being the medicine one of them.
Machine learning has been added to the pipeline of biological studies and, among other important results, it has produced a big impact improving the identification of discriminant pathways, as shown in the study done by Barla et al\cite{Barla2014}.

Machine Learning has been widely used to study different types of cancer.
Some examples of it will be provided in the following section.

%To be extended!
It is also important to dedicate a few words to the Gene Set Enrichment Analysis (GSEA) method \cite{Subramanian15545} which was presented in 2005.
GSEA has had a big impact in the statistical analysis of gene sets with more than 10000 citations.
It is an analytical method that allows the researchers to focus on gene sets instead of individual genes, as it was done before.
Thanks to that, it enables the detection of biological processes like metabolic pathways, transcriptional programs or stress responses. 
Apart from being an statistical analysis method, it also provides a software package an a database composed by more than 1000 gene sets that facilitates its usage and experimentation.

In relation with that methodology, it is also interesting to remark the importance of choosing a proper metric for the ranking of genes, as shown in the work carried out by Zyla et al.\cite{Zyla2017}.

\subsection{Similar work}

As commented before, it is easy to find examples of the usage of machine learning in the field of cancer study. It has been used for different purposes such as classification of tumors, predictions of future tumor development, treatment prediction, and also with the purpose of boosting the performance of the biological analysis pipelines using techniques like feature selection.

% Inside the same field of cancer study, it is easy to find several examples of the usage of machine learning to boost the performance of the pipelines.

An interesting example is found in the study conducted by Ten et al.\cite{Tan2018} where machine learning techniques were introduced in their pipeline in order to improve the analysis of multiple gene expression profiles in cervical cancer.
A particular important fact extracted from that article is that previous studies were focused either in statistical analysis methods or machine learning methods, but that one integrates both methodologies for the meta-analysis, which is also one of the objectives of the presented work.


\subsection{Ongoing Projects}
%To be extended!
% Add ongoing project on mantle cell lymphoma research.