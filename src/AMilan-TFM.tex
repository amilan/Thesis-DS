\documentclass[12pt,a4paper,twoside, openany]{book}
% \documentclass[12pt,a4paper,twoside]{book}
\usepackage{graphicx}
\usepackage{setspace}	%double spacing for text, single for captions, footnotes, etc.
%\usepackage{hypernat} 	%substitut de cite que permet fer hyperlinks
\usepackage{natbib}		% substituye a 'hypernat' que funciona en Windows.
\usepackage[english]{babel}
\usepackage[utf8]{inputenc}
\usepackage{color}
\usepackage{hhline} 		% extended styles for tables
\usepackage{multirow}
\usepackage{subfigure}
\usepackage{acronym}
\usepackage{hyperref}
\usepackage{amsmath,amsmath,amssymb} 
\usepackage{fancyhdr}
\usepackage{epsfig, amsmath}
\usepackage{algorithm}
\usepackage{algorithmic}

% \usepackage{biblatex}

% general settings
\hypersetup{
	linktocpage=true,
	colorlinks=true,
	linkcolor=blue,
	citecolor=blue,
}
\definecolor{Hgray}{gray}{0.6}

\newenvironment{definition}[1][Definition]{\begin{trivlist}
\item[\hskip \labelsep {\bfseries #1}]}{\end{trivlist}}

\setlength{\topmargin}{0cm}
\setlength{\textheight}{23cm}
\setlength{\textwidth}{17cm}
\setlength{\oddsidemargin}{0cm}
\setlength{\evensidemargin}{0cm}
\setlength{\headheight}{1cm}

% indica que las 'sub-sub-sections' sean numeradas y aparezcan en el indice
\setcounter{secnumdepth}{3}
\setcounter{tocdepth}{2}

% settings for code
\renewcommand{\algorithmicrequire}{\textbf{Entrada: }}
\renewcommand{\algorithmicensure}{\textbf{Salida: }}

% \addbibresource{referencias.bib}

%%%%%%%%%%%%
% DOCUMENT %
%%%%%%%%%%%%
\begin{document}

% portada
\newpage
\thispagestyle{empty}

\baselineskip 2em

%\vspace*{1cm}

\centerline{\includegraphics[width=0.6\textwidth]{../images/UOC-logo}}
\begin{center}
\textsc{Universitat Oberta de Catalunya (UOC) \\
 Master's Degree in Data Science (\textit{Data Science})\\}

%\centerline {\pic{UOC}{4cm}}

\vspace*{1.5cm}

\textsc{\Large MASTER'S THESIS}

\vspace*{0.5cm}

\textsc{\large Area: Data Mining and Machine Learning}


%\textbf{\Huge VirtualTechLab Model: }

\vspace*{2.0cm}

\textbf{\Large Study of the Transcriptional Function of Cyclin D1 in Leukemia}

\textbf{\large Comparison of the gene expression regulated by cyclin D1 in lymphomas against gene expression in DNA damage.}

\vspace{2.5cm}
\baselineskip 1em

\baselineskip 2em
-----------------------------------------------------------------------------\\
Author:     Antonio Milán Otero\\
Tutor:      Carles Barceló\\
Professor:  Jordi Casas Roma\\
-----------------------------------------------------------------------------\\
\vspace*{1.5cm}
Barcelona, \today

\end{center}

\newpage
\pagestyle{empty}
\hfill

\newpage
% abstract
\pagenumbering{roman} 
\setcounter{page}{1} 
\pagestyle{plain}

%%%%%%%%%%%%%%%%
%%% CREDITOS %%%
%%%%%%%%%%%%%%%%
\chapter*{License}

% Una página con la especificación de créditos/copyright para el proyecto (ya sea aplicación por un lado y documentación por el otro, o unificadamente), así como la del uso de marcas, productos o servicios de terceros (incluidos códigos fuente). Si una persona diferente al autor colaboró en el proyecto, tiene que quedar explicitada su identidad y qué hizo.

% A continuación se ejemplifica el caso más habitual, aunque se puede modificar por cualquier otra alternativa:

\vspace{1cm}

\begin{figure}[ht]
    \centering
	\includegraphics[scale=1]{../images/license_int.png}
\end{figure}

This work is licensed under a Creative Commons Attribution - NonCommercial - NoDerivatives 4.0 International License.

\href{https://creativecommons.org/licenses/by-nc-nd/4.0/}{Creative Commons 4.0 International License}.

% Esta obra está sujeta a una licencia de Reconocimiento -  NoComercial - SinObraDerivada

% \href{https://creativecommons.org/licenses/by-nc-nd/3.0/es/}{3.0 España de CreativeCommons}.

%%%%%%%%%%%%%
%%% FICHA %%%
%%%%%%%%%%%%%
\chapter*{THESIS INDEX CARD}

\begin{table}[ht]
	\centering{}
	\renewcommand{\arraystretch}{2}
	\begin{tabular}{r | p{8cm}}
		\hline
		Title: & Study of the transcriptional function \newline of Cyclin D1 in Leukemia\\
		\hline
        Author: & Antonio Milán Otero\\
		\hline
        Teacher collaborator: & Carles Barceló\\
		\hline
        Teacher responsible for the subject: & Jordi Casas Roma\\
		\hline
        Date of delivery (mm/aaaa): & 06/2019\\
		\hline
        Degree or program: & MSc in Data Science\\
		\hline
        Thesis area: & Data Mining and Machine Learning\\
		\hline
        Language: & English\\
		\hline
        Keywords & Leukemia, Mantle Cell Lymphoma, Cyclin D1, DNA-damage Repair, Machine Learning\\
		\hline
	\end{tabular}
\end{table}

%%%%%%%%%%%%%%%%%%%
%%% DEDICATORIA %%%
%%%%%%%%%%%%%%%%%%%
\chapter*{Dedication/Quote}

To be completed.

%%%%%%%%%%%%%%%%%%%
%%% Agradecimientos %%%
%%%%%%%%%%%%%%%%%%%
\chapter*{Acknowledgment}

To be completed.

%%%%%%%%%%%%%%%%
%%% RESUMEN  %%%
%%%%%%%%%%%%%%%%
\chapter*{Abstract}
\addcontentsline{toc}{chapter}{Abstract}

\onehalfspacing

%%Motivation
%% Why?
Leukemia is a type of cancer that starts in blood-forming tissue, such as the bone marrow. It causes the production of large numbers of abnormal blood cells that end up entering into the bloodstream. \footnote{https://www.cancer.gov/publications/dictionaries/cancer-terms/def/leukemia}

Within the different types of leukemia, Mantle Cell Lymphoma (MCL) is the one with the worst prognosis due to the short survival average of a patient, which is close to three years.
This tumor is characterized by the over-expression of Cyclin D1, a protein that helps control cell division. MCL is also characterized by the binding of this protein to certain regions of DNA involved in the regulation of DNA-damage response (DDR).

%% Problem statement
%% What problem do we try to solve? 
%% Add the explanation of DDR!!!!

The presented study aims to identify similarities between the gene expression regulated by Cyclin D1 in lymphomas and the gene expression in DNA damage.
That knowledge will allow the exploration of essential mechanisms of carcinogenesis and help in the identification of genes that could be an interesting therapeutic target in the process of tumor progression. Additionally, new biomarkers that could be used in early diagnosis can be found.

%% Approach
%% How did you go about solving or making progress on the problem?
With the addition of Machine Learning algorithms to the biology analysis pipeline, this project explores new ways to improve the traditional methodologies and boost the identification of significantly enriched genes that will serve the purposes mentioned above.

%% Results
%% What's the answer?
%% To be filled lated.

%% Conclusions
%% What are the implications of your answer?
%% To be filled later

\vspace{1.5cm}

\textbf{Keywords}: Leukemia, Mantle Cell Lymphoma, Cyclin D1, DNA-damage Repair, Machine Learning.
\newpage

\pagestyle{fancy}
\renewcommand{\chaptermark}[1]{ \markboth{#1}{}}
\renewcommand{\sectionmark}[1]{\markright{ \thesection.\ #1}}
\lhead[\fancyplain{}{\bfseries\thepage}]{\fancyplain{}{\bfseries\rightmark}}
\rhead[\fancyplain{}{\bfseries\leftmark}]{\fancyplain{}{\bfseries\thepage}}
\cfoot{}

% indice
\cleardoublepage
\phantomsection
\addcontentsline{toc}{chapter}{Index}
\tableofcontents
% listado de figuras
% \cleardoublepage
\clearpage
\phantomsection
\addcontentsline{toc}{chapter}{List of Figures}
\listoffigures
% listado de tablas
% \cleardoublepage
\clearpage
\phantomsection
\addcontentsline{toc}{chapter}{List of Tables}
\listoftables

\thispagestyle{empty}

% This could be used to avoid the list of tables being the 1st page
\newpage
\thispagestyle{empty}
\hfill

\pagenumbering{arabic}

\pagestyle{fancy}
\renewcommand{\chaptermark}[1]{ \markboth{#1}{}}
\renewcommand{\sectionmark}[1]{\markright{ \thesection.\ #1}}
\lhead[\fancyplain{}{\bfseries\thepage}]{\fancyplain{}{\bfseries\rightmark}}
\rhead[\fancyplain{}{\bfseries\leftmark}]{\fancyplain{}{\bfseries\thepage}}
\cfoot{}

\onehalfspacing

% capitulos del documento
\chapter{Introduction}
\label{chapter:introduction}

% \chapter{Introduction}
% \label{chapter:introduction}

%%% SECTION
\section{General description of the problem}

Leukemia is a set of tumor processes that causes an uncontrolled increase in leukocytes (white blood cells) in the blood or lymphatic organs.

Cyclin D1 is an oncogene frequently overexpressed in cancer, especially in Leukemia.
It is known that Cyclin D1 is one of the main regulators of the cell cycle, but its role as a regulator of transcription (the process that generates the proteins needed to control all cellular processes) remains unknown.

It is also well-known that Cyclin D1 binds to the promoter regions of many genes, although the result of its transcriptional activity remains unknown as well. That transcriptional activity is believed to be fundamental in the development of Leukemia.

Recent studies of global transcriptional analysis have generated an enormous amount of data susceptible to be analyzed by data mining and machine learning techniques, being its interpretation fundamental to know the basic mechanisms of the cells that can lead to Leukemia.
Obviously, the generation of new drugs will depend on knowing these processes in detail.

In human cells, both metabolic activities and environmental factors, such as UV rays or radioactivity, can cause DNA damage. Many of these lesions produce potentially harmful mutations in the genome of the cell, which affects the survival of their descendant cells at the time of mitosis or induces malignant processes that end up leading to a tumor.

Several human cancers have been linked to DNA abnormalities such as dislocations, deletions and mutations. The clarification of the mechanisms that initiate the process of repairing DNA damage (DNA-damage response or DDR) will lead to improve the prediction of cancer risk and the treatment in the early stages. More extensive studies of the damage and DNA repair pathways could lead to the development of new therapies aimed at strengthening the natural defense systems of the cells that prevent a tumor from being developed.

Within Leukemia, Mantle Cell Lymphoma (MCL) has the worst prognosis due to the fact that the survival average of patients is close to three years. Identified in the 1990s, it is a difficult disease to diagnose and rarely considered cured. The research to find biomarkers to improve its diagnosis is actively pursued all over the world. This tumor is characterized by the overexpression of Cyclin D1 and the binding of this protein to certain regions of DNA involved in the regulation of DDR.

The project presented here aims to analyze the similarity of the gene expression regulated by Cyclin D1 of MCL with respect to gene expression of DDR. This would allow exploring possible essential mechanisms of carcinogenesis and focus on the genes that could be interesting therapeutic targets in the process of tumor progression. In addition, new biomarkers could be used to help in an early diagnosis, often linked to a better survival rate.

The data sets published for MCL (GSE21452 \cite{mclData:2011}) and DDR (GSE25848 \cite{ddrData:2011}) will be used to generate a gene signature in which the significantly enriched genes will be identified in order to study their possible role as a therapeutic target and as a biomarker.

The \textit{R} environment will be used to align the reads, generate quality controls and finally generate the gene signature through Gene Set Enrichment Analysis (GSEA). All this process will be boosted with the addition of machine learning methodologies.


\newpage

% \chapter{Motivation}
% \label{chapter:motivation}

%%% SECTION
\section{Motivation}

\subsection{Why this project?}
The amount of data collected in scientific researches has increased exponentially during the last decades, making the usage of data science methodologies a good fit for improving the final analysis and results.
This project is a clear example of how the advances in data science can trigger new ways of doing science, expanding the existing tools in order to achieve better results.

\subsection{What can I add?}
During the course of my master in data science, I have been learning about all the different aspects of a data science project, starting from a project management point of view and continuing with all the different phases of acquisition, storage, hypothesis and modeling, visualization and deployment.
For this specific project, though, my focus will be in the area of data mining and machine learning, and that is what I think I can add to the project, my accumulated experience in the commented area.

\subsection{Personal interest}
My personal interest in this project comes from the fact, or bad luck, of having close family and friends affected for Leukemia, therefore, as soon as I saw the proposal of this project I felt emotionally connected to it.

Apart from that first reason, I also consider that one of the best usages of the advances of data mining and machine learning is to help in the creation of a better society, being one of its strongest foundations the improvement of the quality of the health of each individual. Therefore, I feel responsible of using my new acquired knowledge in areas that can lead to that goal.

\subsection{How can this project improve my CV?}
At the time of writing this, I am working as a research engineer in the control software group in MAX IV Laboratory, a synchrotron located in the south of Sweden that has the purpose of improving the scientific researches in a global encompass.
Until now, my main activities has been related to the build of software for all the different aspects of the control system, from the very low level control, writing drivers for equipments, up to the high level software like graphical user interfaces that enable the scientist to perform their jobs, passing through all the layers in between, like software libraries, servers, etc.
In other words, I have been always close to the control, synchronization and data acquisition, with this project, I can expand my coverage and help also in the next phase of a scientific research, the analysis of the generated data.


\newpage
% \chapter{Introduction}
% \label{chapter:introduction}

%%% SECTION
\section{Project Objectives}
\subsection{Primary}
Analysis of the similarities of the gene expression regulated by Cyclin D1 in Mantle Cell Lymphoma (MCL) and the gene expression of the DNA-damage response (DDR) using Machine Learning.

\subsection{Secondaries}
To generate a gene signature in which the significantly enriched genes are identified in order to study its possible role as a potential therapeutic target as a biomarker.


\newpage
\section{Description of the Methodology Used}

\newpage
\section{Project Research Plan}
\chapter{Literature Review}
\label{chapter:literature}

% \chapter{Literature Review}
% \label{chapter:literature}

%%% SECTION
\section{Literature Review}
\subsection{Foundations}

The overexpression of cyclin D1 in human cancer is well-known\cite{Lamb2003} and has been reported in several studies\cite{2017Reena}.
An interesting recent work conducted by Albero et al.\cite{10.1172/JCI96520}, focuses on the study of this overexpression and how it produces a global trancriptional donwmodulation in lymphoid neoplasms. In their own words:

\textit{This finding of global transcriptional dysregulation expands the known functions of oncogenic cyclin D1 and suggests the therapeutic potential of targeting the transcriptional machinery in cyclin D1–overexpressing tumors.}
\cite{10.1172/JCI96520}

In parallel, other studies \cite{DiSante2017} \cite{Casimiro2016}, have shown the role of cyclin D1 in the cell cycle and its influence in the DNA-damage repair process.

This two concepts have triggered the main idea behind the objectives of the project presented here.

In addition to the already commented works, the field of artificial intelligence and in particular the Machine Learning discipline inside of it, has been winning attention in many fields, being the medicine one of them.
Machine Learning has been widely used to study different types of cancer, and some examples of it will be provided in the following section.


As a result, Machine Learning has been added to the pipeline of biological studies and, among other important consequences, it has produced a big impact improving the identification of discriminant pathways, as shown in the study done by Barla et al\cite{Barla2014}.

%To be extended!
Another fundamental method for this project is the Gene Set Enrichment Analysis (GSEA) \cite{Subramanian15545} which was presented in 2005.
GSEA has had a big impact in the statistical analysis of gene sets with more than 10000 citations.
It is an analytical method that allows the researchers to focus on gene sets instead of individual genes, as it was done before.
Thanks to that, it enables the detection of biological processes like metabolic pathways, transcriptional programs or stress responses. 
Apart from being a statistical analysis method, it also provides a software package and a database composed by more than 1000 gene sets that facilitates its usage and experimentation.

In relation with that methodology, it is also interesting to remark the importance of choosing a proper metric for the ranking of genes, as shown in the work carried out by Zyla et al.\cite{Zyla2017}.

\subsection{Similar Work}

As commented before, it is easy to find examples of the usage of Machine Learning in the field of cancer study. It has been widely used for different purposes such as classification\cite{Chuang2007} and prediction of tumors, treatment prediction, and also to boost the performance of the biological analysis pipelines using techniques like feature selection\cite{SINGH201552}\cite{Bashiri2017}.

% Inside the same field of cancer study, it is easy to find several examples of the usage of machine learning to boost the performance of the pipelines.

An interesting example is found in the study conducted by Ten et al.\cite{Tan2018} where Machine Learning techniques were introduced in their pipeline in order to improve the analysis of multiple gene expression profiles in cervical cancer.
A particular important fact extracted from that article, is that previous studies were focused either in statistical analysis methods or Machine Learning methods, but that one integrates both methodologies for the meta-analysis, which is also one of the objectives of the presented work.

Another similar interesting work is found in the study elaborated by Park et al.\cite{Park2018}. In there, the identification of disease-related genes and disease mechanism is investigated using Machine Learning techniques. The study presents a novel method for gene-gene interaction (GGI) based on the usage of the Random Forest algorithm. This method is suitable for the discovery of significant GGI from heterogeneous gene expression datasets, and has the potential to be used in the research of different disease groups.

\subsection{Ongoing and Future Projects}
%To be extended!
% Add ongoing project on mantle cell lymphoma research.

The last section of this chapter is dedicated to the ongoing and future studies and developments related with Mantle Cell Lymphoma.

Several studies\cite{Steiner2018}\cite{Schieber2018}\cite{Dreyling2016} agree on the necessity of the development of more personalized (patient-centric) treatments. Such treatments will be possible through an evaluation of each patient unique set of genomic complications. But before this point will be reached, new biomarkers and pathways that will enhance the understanding of MCL need to be discovered. This is an active area of study and it is also one of the purposes of the work presented here.

\chapter{Literature Review}
\label{chapter:literature}

% \chapter{Literature Review}
% \label{chapter:literature}

%%% SECTION
\section{Literature Review}
\subsection{Foundations}

The overexpression of cyclin D1 in human cancer is well-known\cite{Lamb2003} and has been reported in several studies\cite{2017Reena}.
An interesting recent work conducted by Albero et al.\cite{10.1172/JCI96520}, focuses on the study of this overexpression and how it produces a global trancriptional donwmodulation in lymphoid neoplasms. In their own words:

\textit{This finding of global transcriptional dysregulation expands the known functions of oncogenic cyclin D1 and suggests the therapeutic potential of targeting the transcriptional machinery in cyclin D1–overexpressing tumors.}
\cite{10.1172/JCI96520}

In parallel, other studies \cite{DiSante2017} \cite{Casimiro2016}, have shown the role of cyclin D1 in the cell cycle and its influence in the DNA-damage repair process.

This two concepts have triggered the main idea behind the objectives of the project presented here.

In addition to the already commented works, the field of artificial intelligence and in particular the Machine Learning discipline inside of it, has been winning attention in many fields, being the medicine one of them.
Machine Learning has been widely used to study different types of cancer, and some examples of it will be provided in the following section.


As a result, Machine Learning has been added to the pipeline of biological studies and, among other important consequences, it has produced a big impact improving the identification of discriminant pathways, as shown in the study done by Barla et al\cite{Barla2014}.

%To be extended!
Another fundamental method for this project is the Gene Set Enrichment Analysis (GSEA) \cite{Subramanian15545} which was presented in 2005.
GSEA has had a big impact in the statistical analysis of gene sets with more than 10000 citations.
It is an analytical method that allows the researchers to focus on gene sets instead of individual genes, as it was done before.
Thanks to that, it enables the detection of biological processes like metabolic pathways, transcriptional programs or stress responses. 
Apart from being a statistical analysis method, it also provides a software package and a database composed by more than 1000 gene sets that facilitates its usage and experimentation.

In relation with that methodology, it is also interesting to remark the importance of choosing a proper metric for the ranking of genes, as shown in the work carried out by Zyla et al.\cite{Zyla2017}.

\subsection{Similar Work}

As commented before, it is easy to find examples of the usage of Machine Learning in the field of cancer study. It has been widely used for different purposes such as classification\cite{Chuang2007} and prediction of tumors, treatment prediction, and also to boost the performance of the biological analysis pipelines using techniques like feature selection\cite{SINGH201552}\cite{Bashiri2017}.

% Inside the same field of cancer study, it is easy to find several examples of the usage of machine learning to boost the performance of the pipelines.

An interesting example is found in the study conducted by Ten et al.\cite{Tan2018} where Machine Learning techniques were introduced in their pipeline in order to improve the analysis of multiple gene expression profiles in cervical cancer.
A particular important fact extracted from that article, is that previous studies were focused either in statistical analysis methods or Machine Learning methods, but that one integrates both methodologies for the meta-analysis, which is also one of the objectives of the presented work.

Another similar interesting work is found in the study elaborated by Park et al.\cite{Park2018}. In there, the identification of disease-related genes and disease mechanism is investigated using Machine Learning techniques. The study presents a novel method for gene-gene interaction (GGI) based on the usage of the Random Forest algorithm. This method is suitable for the discovery of significant GGI from heterogeneous gene expression datasets, and has the potential to be used in the research of different disease groups.

\subsection{Ongoing and Future Projects}
%To be extended!
% Add ongoing project on mantle cell lymphoma research.

The last section of this chapter is dedicated to the ongoing and future studies and developments related with Mantle Cell Lymphoma.

Several studies\cite{Steiner2018}\cite{Schieber2018}\cite{Dreyling2016} agree on the necessity of the development of more personalized (patient-centric) treatments. Such treatments will be possible through an evaluation of each patient unique set of genomic complications. But before this point will be reached, new biomarkers and pathways that will enhance the understanding of MCL need to be discovered. This is an active area of study and it is also one of the purposes of the work presented here.

\chapter{Literature Review}
\label{chapter:literature}

% \chapter{Literature Review}
% \label{chapter:literature}

%%% SECTION
\section{Literature Review}
\subsection{Foundations}

The overexpression of cyclin D1 in human cancer is well-known\cite{Lamb2003} and has been reported in several studies\cite{2017Reena}.
An interesting recent work conducted by Albero et al.\cite{10.1172/JCI96520}, focuses on the study of this overexpression and how it produces a global trancriptional donwmodulation in lymphoid neoplasms. In their own words:

\textit{This finding of global transcriptional dysregulation expands the known functions of oncogenic cyclin D1 and suggests the therapeutic potential of targeting the transcriptional machinery in cyclin D1–overexpressing tumors.}
\cite{10.1172/JCI96520}

In parallel, other studies \cite{DiSante2017} \cite{Casimiro2016}, have shown the role of cyclin D1 in the cell cycle and its influence in the DNA-damage repair process.

This two concepts have triggered the main idea behind the objectives of the project presented here.

In addition to the already commented works, the field of artificial intelligence and in particular the Machine Learning discipline inside of it, has been winning attention in many fields, being the medicine one of them.
Machine Learning has been widely used to study different types of cancer, and some examples of it will be provided in the following section.


As a result, Machine Learning has been added to the pipeline of biological studies and, among other important consequences, it has produced a big impact improving the identification of discriminant pathways, as shown in the study done by Barla et al\cite{Barla2014}.

%To be extended!
Another fundamental method for this project is the Gene Set Enrichment Analysis (GSEA) \cite{Subramanian15545} which was presented in 2005.
GSEA has had a big impact in the statistical analysis of gene sets with more than 10000 citations.
It is an analytical method that allows the researchers to focus on gene sets instead of individual genes, as it was done before.
Thanks to that, it enables the detection of biological processes like metabolic pathways, transcriptional programs or stress responses. 
Apart from being a statistical analysis method, it also provides a software package and a database composed by more than 1000 gene sets that facilitates its usage and experimentation.

In relation with that methodology, it is also interesting to remark the importance of choosing a proper metric for the ranking of genes, as shown in the work carried out by Zyla et al.\cite{Zyla2017}.

\subsection{Similar Work}

As commented before, it is easy to find examples of the usage of Machine Learning in the field of cancer study. It has been widely used for different purposes such as classification\cite{Chuang2007} and prediction of tumors, treatment prediction, and also to boost the performance of the biological analysis pipelines using techniques like feature selection\cite{SINGH201552}\cite{Bashiri2017}.

% Inside the same field of cancer study, it is easy to find several examples of the usage of machine learning to boost the performance of the pipelines.

An interesting example is found in the study conducted by Ten et al.\cite{Tan2018} where Machine Learning techniques were introduced in their pipeline in order to improve the analysis of multiple gene expression profiles in cervical cancer.
A particular important fact extracted from that article, is that previous studies were focused either in statistical analysis methods or Machine Learning methods, but that one integrates both methodologies for the meta-analysis, which is also one of the objectives of the presented work.

Another similar interesting work is found in the study elaborated by Park et al.\cite{Park2018}. In there, the identification of disease-related genes and disease mechanism is investigated using Machine Learning techniques. The study presents a novel method for gene-gene interaction (GGI) based on the usage of the Random Forest algorithm. This method is suitable for the discovery of significant GGI from heterogeneous gene expression datasets, and has the potential to be used in the research of different disease groups.

\subsection{Ongoing and Future Projects}
%To be extended!
% Add ongoing project on mantle cell lymphoma research.

The last section of this chapter is dedicated to the ongoing and future studies and developments related with Mantle Cell Lymphoma.

Several studies\cite{Steiner2018}\cite{Schieber2018}\cite{Dreyling2016} agree on the necessity of the development of more personalized (patient-centric) treatments. Such treatments will be possible through an evaluation of each patient unique set of genomic complications. But before this point will be reached, new biomarkers and pathways that will enhance the understanding of MCL need to be discovered. This is an active area of study and it is also one of the purposes of the work presented here.

\chapter{Literature Review}
\label{chapter:literature}

% \chapter{Literature Review}
% \label{chapter:literature}

%%% SECTION
\section{Literature Review}
\subsection{Foundations}

The overexpression of cyclin D1 in human cancer is well-known\cite{Lamb2003} and has been reported in several studies\cite{2017Reena}.
An interesting recent work conducted by Albero et al.\cite{10.1172/JCI96520}, focuses on the study of this overexpression and how it produces a global trancriptional donwmodulation in lymphoid neoplasms. In their own words:

\textit{This finding of global transcriptional dysregulation expands the known functions of oncogenic cyclin D1 and suggests the therapeutic potential of targeting the transcriptional machinery in cyclin D1–overexpressing tumors.}
\cite{10.1172/JCI96520}

In parallel, other studies \cite{DiSante2017} \cite{Casimiro2016}, have shown the role of cyclin D1 in the cell cycle and its influence in the DNA-damage repair process.

This two concepts have triggered the main idea behind the objectives of the project presented here.

In addition to the already commented works, the field of artificial intelligence and in particular the Machine Learning discipline inside of it, has been winning attention in many fields, being the medicine one of them.
Machine Learning has been widely used to study different types of cancer, and some examples of it will be provided in the following section.


As a result, Machine Learning has been added to the pipeline of biological studies and, among other important consequences, it has produced a big impact improving the identification of discriminant pathways, as shown in the study done by Barla et al\cite{Barla2014}.

%To be extended!
Another fundamental method for this project is the Gene Set Enrichment Analysis (GSEA) \cite{Subramanian15545} which was presented in 2005.
GSEA has had a big impact in the statistical analysis of gene sets with more than 10000 citations.
It is an analytical method that allows the researchers to focus on gene sets instead of individual genes, as it was done before.
Thanks to that, it enables the detection of biological processes like metabolic pathways, transcriptional programs or stress responses. 
Apart from being a statistical analysis method, it also provides a software package and a database composed by more than 1000 gene sets that facilitates its usage and experimentation.

In relation with that methodology, it is also interesting to remark the importance of choosing a proper metric for the ranking of genes, as shown in the work carried out by Zyla et al.\cite{Zyla2017}.

\subsection{Similar Work}

As commented before, it is easy to find examples of the usage of Machine Learning in the field of cancer study. It has been widely used for different purposes such as classification\cite{Chuang2007} and prediction of tumors, treatment prediction, and also to boost the performance of the biological analysis pipelines using techniques like feature selection\cite{SINGH201552}\cite{Bashiri2017}.

% Inside the same field of cancer study, it is easy to find several examples of the usage of machine learning to boost the performance of the pipelines.

An interesting example is found in the study conducted by Ten et al.\cite{Tan2018} where Machine Learning techniques were introduced in their pipeline in order to improve the analysis of multiple gene expression profiles in cervical cancer.
A particular important fact extracted from that article, is that previous studies were focused either in statistical analysis methods or Machine Learning methods, but that one integrates both methodologies for the meta-analysis, which is also one of the objectives of the presented work.

Another similar interesting work is found in the study elaborated by Park et al.\cite{Park2018}. In there, the identification of disease-related genes and disease mechanism is investigated using Machine Learning techniques. The study presents a novel method for gene-gene interaction (GGI) based on the usage of the Random Forest algorithm. This method is suitable for the discovery of significant GGI from heterogeneous gene expression datasets, and has the potential to be used in the research of different disease groups.

\subsection{Ongoing and Future Projects}
%To be extended!
% Add ongoing project on mantle cell lymphoma research.

The last section of this chapter is dedicated to the ongoing and future studies and developments related with Mantle Cell Lymphoma.

Several studies\cite{Steiner2018}\cite{Schieber2018}\cite{Dreyling2016} agree on the necessity of the development of more personalized (patient-centric) treatments. Such treatments will be possible through an evaluation of each patient unique set of genomic complications. But before this point will be reached, new biomarkers and pathways that will enhance the understanding of MCL need to be discovered. This is an active area of study and it is also one of the purposes of the work presented here.

\chapter{Literature Review}
\label{chapter:literature}

% \chapter{Literature Review}
% \label{chapter:literature}

%%% SECTION
\section{Literature Review}
\subsection{Foundations}

The overexpression of cyclin D1 in human cancer is well-known\cite{Lamb2003} and has been reported in several studies\cite{2017Reena}.
An interesting recent work conducted by Albero et al.\cite{10.1172/JCI96520}, focuses on the study of this overexpression and how it produces a global trancriptional donwmodulation in lymphoid neoplasms. In their own words:

\textit{This finding of global transcriptional dysregulation expands the known functions of oncogenic cyclin D1 and suggests the therapeutic potential of targeting the transcriptional machinery in cyclin D1–overexpressing tumors.}
\cite{10.1172/JCI96520}

In parallel, other studies \cite{DiSante2017} \cite{Casimiro2016}, have shown the role of cyclin D1 in the cell cycle and its influence in the DNA-damage repair process.

This two concepts have triggered the main idea behind the objectives of the project presented here.

In addition to the already commented works, the field of artificial intelligence and in particular the Machine Learning discipline inside of it, has been winning attention in many fields, being the medicine one of them.
Machine Learning has been widely used to study different types of cancer, and some examples of it will be provided in the following section.


As a result, Machine Learning has been added to the pipeline of biological studies and, among other important consequences, it has produced a big impact improving the identification of discriminant pathways, as shown in the study done by Barla et al\cite{Barla2014}.

%To be extended!
Another fundamental method for this project is the Gene Set Enrichment Analysis (GSEA) \cite{Subramanian15545} which was presented in 2005.
GSEA has had a big impact in the statistical analysis of gene sets with more than 10000 citations.
It is an analytical method that allows the researchers to focus on gene sets instead of individual genes, as it was done before.
Thanks to that, it enables the detection of biological processes like metabolic pathways, transcriptional programs or stress responses. 
Apart from being a statistical analysis method, it also provides a software package and a database composed by more than 1000 gene sets that facilitates its usage and experimentation.

In relation with that methodology, it is also interesting to remark the importance of choosing a proper metric for the ranking of genes, as shown in the work carried out by Zyla et al.\cite{Zyla2017}.

\subsection{Similar Work}

As commented before, it is easy to find examples of the usage of Machine Learning in the field of cancer study. It has been widely used for different purposes such as classification\cite{Chuang2007} and prediction of tumors, treatment prediction, and also to boost the performance of the biological analysis pipelines using techniques like feature selection\cite{SINGH201552}\cite{Bashiri2017}.

% Inside the same field of cancer study, it is easy to find several examples of the usage of machine learning to boost the performance of the pipelines.

An interesting example is found in the study conducted by Ten et al.\cite{Tan2018} where Machine Learning techniques were introduced in their pipeline in order to improve the analysis of multiple gene expression profiles in cervical cancer.
A particular important fact extracted from that article, is that previous studies were focused either in statistical analysis methods or Machine Learning methods, but that one integrates both methodologies for the meta-analysis, which is also one of the objectives of the presented work.

Another similar interesting work is found in the study elaborated by Park et al.\cite{Park2018}. In there, the identification of disease-related genes and disease mechanism is investigated using Machine Learning techniques. The study presents a novel method for gene-gene interaction (GGI) based on the usage of the Random Forest algorithm. This method is suitable for the discovery of significant GGI from heterogeneous gene expression datasets, and has the potential to be used in the research of different disease groups.

\subsection{Ongoing and Future Projects}
%To be extended!
% Add ongoing project on mantle cell lymphoma research.

The last section of this chapter is dedicated to the ongoing and future studies and developments related with Mantle Cell Lymphoma.

Several studies\cite{Steiner2018}\cite{Schieber2018}\cite{Dreyling2016} agree on the necessity of the development of more personalized (patient-centric) treatments. Such treatments will be possible through an evaluation of each patient unique set of genomic complications. But before this point will be reached, new biomarkers and pathways that will enhance the understanding of MCL need to be discovered. This is an active area of study and it is also one of the purposes of the work presented here.



% bibliografia
\addcontentsline{toc}{chapter}{Bibliography}
% \bibliographystyle{plain}
\bibliographystyle{unsrt}
\bibliography{referencias}
% \nocite{*}

\chapter{Literature Review}
\label{chapter:literature}

% \chapter{Literature Review}
% \label{chapter:literature}

%%% SECTION
\section{Literature Review}
\subsection{Foundations}

The overexpression of cyclin D1 in human cancer is well-known\cite{Lamb2003} and has been reported in several studies\cite{2017Reena}.
An interesting recent work conducted by Albero et al.\cite{10.1172/JCI96520}, focuses on the study of this overexpression and how it produces a global trancriptional donwmodulation in lymphoid neoplasms. In their own words:

\textit{This finding of global transcriptional dysregulation expands the known functions of oncogenic cyclin D1 and suggests the therapeutic potential of targeting the transcriptional machinery in cyclin D1–overexpressing tumors.}
\cite{10.1172/JCI96520}

In parallel, other studies \cite{DiSante2017} \cite{Casimiro2016}, have shown the role of cyclin D1 in the cell cycle and its influence in the DNA-damage repair process.

This two concepts have triggered the main idea behind the objectives of the project presented here.

In addition to the already commented works, the field of artificial intelligence and in particular the Machine Learning discipline inside of it, has been winning attention in many fields, being the medicine one of them.
Machine Learning has been widely used to study different types of cancer, and some examples of it will be provided in the following section.


As a result, Machine Learning has been added to the pipeline of biological studies and, among other important consequences, it has produced a big impact improving the identification of discriminant pathways, as shown in the study done by Barla et al\cite{Barla2014}.

%To be extended!
Another fundamental method for this project is the Gene Set Enrichment Analysis (GSEA) \cite{Subramanian15545} which was presented in 2005.
GSEA has had a big impact in the statistical analysis of gene sets with more than 10000 citations.
It is an analytical method that allows the researchers to focus on gene sets instead of individual genes, as it was done before.
Thanks to that, it enables the detection of biological processes like metabolic pathways, transcriptional programs or stress responses. 
Apart from being a statistical analysis method, it also provides a software package and a database composed by more than 1000 gene sets that facilitates its usage and experimentation.

In relation with that methodology, it is also interesting to remark the importance of choosing a proper metric for the ranking of genes, as shown in the work carried out by Zyla et al.\cite{Zyla2017}.

\subsection{Similar Work}

As commented before, it is easy to find examples of the usage of Machine Learning in the field of cancer study. It has been widely used for different purposes such as classification\cite{Chuang2007} and prediction of tumors, treatment prediction, and also to boost the performance of the biological analysis pipelines using techniques like feature selection\cite{SINGH201552}\cite{Bashiri2017}.

% Inside the same field of cancer study, it is easy to find several examples of the usage of machine learning to boost the performance of the pipelines.

An interesting example is found in the study conducted by Ten et al.\cite{Tan2018} where Machine Learning techniques were introduced in their pipeline in order to improve the analysis of multiple gene expression profiles in cervical cancer.
A particular important fact extracted from that article, is that previous studies were focused either in statistical analysis methods or Machine Learning methods, but that one integrates both methodologies for the meta-analysis, which is also one of the objectives of the presented work.

Another similar interesting work is found in the study elaborated by Park et al.\cite{Park2018}. In there, the identification of disease-related genes and disease mechanism is investigated using Machine Learning techniques. The study presents a novel method for gene-gene interaction (GGI) based on the usage of the Random Forest algorithm. This method is suitable for the discovery of significant GGI from heterogeneous gene expression datasets, and has the potential to be used in the research of different disease groups.

\subsection{Ongoing and Future Projects}
%To be extended!
% Add ongoing project on mantle cell lymphoma research.

The last section of this chapter is dedicated to the ongoing and future studies and developments related with Mantle Cell Lymphoma.

Several studies\cite{Steiner2018}\cite{Schieber2018}\cite{Dreyling2016} agree on the necessity of the development of more personalized (patient-centric) treatments. Such treatments will be possible through an evaluation of each patient unique set of genomic complications. But before this point will be reached, new biomarkers and pathways that will enhance the understanding of MCL need to be discovered. This is an active area of study and it is also one of the purposes of the work presented here.


% \input{extra_refs.tex}

% \printbibliography[Title=References, subtype=online]
% \printbibliography[title=Referemces]

\end{document}