\onehalfspacing
% \chapter{Motivation}
% \label{chapter:motivation}

%%% SECTION
\section{Motivation}

\subsection{Why this project?}
The amount of data collected in scientific researches has increased exponentially during the last decades, making the usage of Data Science methodologies a good fit for improving the final analysis and results.
This project is a clear example of how the advances in Data Science can trigger new ways of doing science, expanding the existing tools in order to achieve better results.

\subsection{What can I add?}
During my MSc in Data Science, I have been learning about all the different aspects of a Data Science project, starting from a project management point of view and continuing with all the different phases of acquisition, storage, hypothesis and modeling, visualization and deployment.
For this specific project, though, my focus will be in the area of Data Mining and Machine Learning, and that is what I think I can add to the project, my accumulated experience in the commented area.

\subsection{Personal interest}
My personal interest in this project comes from the fact, or bad luck, of having close family and friends affected for leukemia, therefore, as soon as I saw the proposal of this project I felt emotionally connected to it.

Apart from that first reason, I also consider that one of the best usages of the advances of Data Mining and Machine Learning is to help in the creation of a better society, being one of its strongest foundations the improvement of the quality of the health of each individual. Therefore, I feel responsible for using my new acquired knowledge in areas that can lead to that goal.

\subsection{How can this project improve my CV?}
At the time of writing this, I am working as a research engineer in the control software group in MAX IV Laboratory, a synchrotron located in the south of Sweden that has the purpose of improving the scientific researches in a global encompass.
Until now, my main activities has been related to the build of software for all the different aspects of the control system, from the very low level control, writing drivers for equipment, up to the high level software like graphical user interfaces that enable the scientist to perform their jobs, passing through all the layers in between, like software libraries, servers, etc.
In other words, I have been always close to the control, synchronization and data acquisition, with this project, I can expand my coverage and help also in the next phase of a scientific research, the analysis of the generated data.
