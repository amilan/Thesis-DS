% \chapter{Introduction}
% \label{chapter:introduction}

%%% SECTION
\section{General description of the problem}

Leukemia is a set of tumor processes that causes an uncontrolled increase in leukocytes (white blood cells) in the blood or lymphatic organs. Cyclin D1 is an oncogene frequently overexpressed in cancer, especially in Leukemia.
It is known that Cyclin D1 is one of the main regulators of the cell cycle, but its role as a regulator of transcription (the process by which the proteins needed to control all cellular processes are generated) remains unknown.
It is also well known that Cyclin D1 binds to the promoter regions of many genes, although the result of its transcriptional activity, that is believed to be fundamental in the development of leukemia, remains unknown as well.
Recent studies of global transcriptional analysis have generated an enormous amount of data susceptible to be analyzed by data mining and machine learning techniques, being its interpretation fundamental to know the basic mechanisms of the cells that can lead to Leukemia.
Obviously, the generation of new drugs will depend on knowing these processes in detail.

In this project, a comparison between the gene expression derived from the regulation done by the Cyclin D1 in Lymphoma and the gene expression in the case of DNA damage will be done. This would allow to explore possible essential mechanisms of \textbf{\textit{carcinogenesis}} and focus on the genes that could really be interesting as a therapeutic targets in the process of tumor progression. In addition, new biomarkers could be used to help in an early diagnosis, often linked to a better survival rate.

In human cells, both metabolic activities and environmental factors, such as UV rays or radioactivity, can cause DNA damage. Many of these lesions produce potentially harmful mutations in the genome of the cell, which affects the survival of their "daughter cells" at the time of mitosis or induces malignant processes that ended up leading to a tumor. Several human cancers have been linked to DNA abnormalities such as dislocations, deletions and mutations. The clarification of the mechanisms that initiate the process of repairing DNA damage (DNA-damage response or DDR) will lead to improve the prediction of cancer risk and treatment in the early stages. More extensive studies of the damage and DNA repair pathways could lead to the development of new therapies aimed at strengthening the natural defense systems of the cells that prevent a tumor from being developed.

Within leukemia, Mantle Cell Lymphoma (MCL) has the worst prognosis since the survival average of patients is close to 3 years. Identified in the 1990s, it is a difficult disease to diagnose and rarely considered cured. The research to find biomarkers to improve its diagnosis is actively pursued all over the world. This tumor is characterized by the overexpression of cyclin D1 and the binding of this protein to certain regions of DNA involved in the regulation of DDR.

The project presented here aims to analyze the similarity of the gene expression regulated by Cyclin D1 of MCL with respect to gene expression of DDR. The data sets published for MCL (GSE21452) and DDR (GSE25848) will be used to generate a gene signature in which the genes "significantly enriched" are identified in order to study their possible role as a possible therapeutic target and as a biomarker. The "R" environment will be used to align the reads, generate quality controls and finally generate the gene signature through Gene Set Enrichment Analysis (GSEA).

----


It must be taken into account that it is a real biomedical research project with great applicability in the treatment of tumors and a very high social return. However, it is not free of uncertainty regarding the possibility of obtaining clear and conclusive results already, as it happens in any type of scientific research, it has never been done before.
