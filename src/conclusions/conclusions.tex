% \chapter{Conclusions}
% \label{chapter:conclusions}

%%% SECTION
\section{Conclusions}

The first important observation from the data extracted from GSEA is the up-regulation of hypoxia gene sets in more than one collection, i.e. in H and C2.

Hypoxia is a condition where low levels of oxygen are supplied to a cell tissue. It is used in cancer treatment to predict the response of a tumor to an specific treatment and it is associated to the resistance of a therapy. Studies like the one conducted by Possik et al. \cite{Possik2014} show that hypoxia sensitizes melanomas to targeted inhibition of the DDR, contributing like that to the tumor expansion.
This up-regulation of the hypoxia gene sets shows an influence of CCND1 on this condition for MCL and DDR.

Another interesting observation is the up-regulation of Apoptosis. Apoptosis is a series of molecular steps that ends up in leading the cell to its death. This process is used by the body to eliminate abnormal or unnecessary cells. This process may be blocked by cancer cells.
The finding of up-regulation on these gene set suggest that CCND1 has a role in the apoptosis process on MCL and DDR and it could be considered for future treatments.

Continuing with the up-regulated gene sets found in this study, GSEA reports an up-regulation of the Notch signaling pathway. As seen in the study conducted by Li et al. \cite{Li2017}, Notch signaling plays a critical role in the development of different forms of cancer, and due to its importance in tumorigenesis and metastasis, blocking Notch signaling pathway may be considered as a potential therapy for cancer treatment, and by extension MCL.

The next up-regulated gene set that worth the attention of this study is the regulation of cell cycle and specially the regulation of mitotic cell cycle. This process consists of a series of steps where chromosomes and other cell materials are duplicated for its posterior usage on the split of the cell into two daughter cells.
The found influence of CCND1 in this process for MCL and DDR could be suggested as a target for further study, with the aim to determine if a possible therapy can be obtained. This idea is inline with the concluded in the study done by Bakhoum et al.: \textit{Cancer cells coopt the mitotic DNA damage response to further propagate chromosomal instability. This offers untapped therapeutic opportunities to target genomic instability in cancer.}\cite{Bakhoum2017}


As a final conclusion, it is worth to remind that the main objective of this study was to analyze the similarities between the gene expressions regulated by Cyclin D1 in MCL and DDR using Machine Learning, that objective has been accomplished through an integrative pipeline where classical statistical methods used in biology has been combined with Data Mining and Machine Learning techniques.

